
\subsection{Redesign strategy and suggestions}

Based on the results gathered both from the inspection and from the user testing, it was possible to define a Redesign Strategy that focused on addressing the three main clusters of problems detected:

\begin{itemize}
    \item \textbf{Information Architecture}, mainly identified during the user testing
    \item \textbf{Inconsistency and Poor Visual Hierarchy}, mainly identified during the inspection
    \item \textbf{Overload of Information}, identified both during the inspection and the user testing
\end{itemize}

The team decided to give a higher priority to problems found or confirmed during the user testing, as usually with this methodology testers spot more meaningful issues that can compromise the navigation and the usability more harshly. For this reason, the team agreed on developing a strategy that first addresses problems connected to information architecture and information overload (first level priority) and then inconsistency and poor hierarchy (second level priority).\\

Regarding information architecture, it is essential to engage in a redesign process that focuses on establishing a logical framework that builds trust in users. In a website such as the one of UNICEF, keeping a hierarchical organisation system can be the right choice, so that the user can understand the different importance and purpose of each page. A key aspect to address is avoiding repetition in the main menu and identifying the links that redirect to another website. The labelling system should be revised: labels should help the website to communicate efficiently, but looking at the testing results it is possible to state how this is not really working at the moment. In fact, users had difficulty in finding the correct page or section quickly, meaning that the language or the sections are not so intuitive. Moreover, it is essential to create consistency between the labelling present in the menu and in the pages. Search systems such as the search engine and the filters are present in the website, but they should be designed in a clearer way in order to be more visible and to convey the information in the right way. Lastly, navigation results to be cumbersome in many cases due to the difficulties in going back and forth between pages and sub-pages. This is exacerbated by the lack of breadcrumbs, which results in getting lost in the website.

To build a better information architecture and improve navigability, the team suggest starting with performing an open and closed card sorting and subsequently a tree testing, in order to understand users’ needs, expectations and mental models.\\

Difficulties in finding information due to a confusing information architecture are amplified by a constant overload of information across the whole website. An effective redesign strategy should take into consideration the improvement of how information is displayed and conveyed, in order to ensure that just the right amount of information appears on each page, enough to make it relevant but not so much that it causes information overload.
The amount of information present in each page of UNICEF’s website prevents users from successfully making decisions, as well as reducing their attention and increasing their cognitive effort. In this case, information of minor importance could be displayed in a drop-down menu or generally being accessible only if the user wants to. In the same way, resources such as news, articles or reports could be condensed at the end of the page, organised in a clear and engaging way. If the navigation system is improved, then it would also be possible to organise contents between pages and sub-pages in a more effective way, allowing users to freely navigate from one content to another one related, following a solid logic.\\

Poor visual hierarchy is undoubtedly connected to the overload of information, but in this case it is more linked to the visual style of elements. By redesigning the sizes of elements (text, cards, images, buttons) to portray their actual relevance, it should also be easier to have a clearer understanding of the purpose of each page and avoid users being paralyzed without knowing what to do. At the moment, since every element appears important, nothing seems important. Some suggestions are to use alignment and composition to create focal points, as well as using reading patterns to easily convey most important information. Moreover, correct sizing and engaging colours should be used to create visual impact and to draw attention to specific elements. To avoid bombarding the users with information, white spaces are essential: they maintain balance and help in emphasising elements on the page. Lastly, a redesign of the font hierarchy is considered to be necessary, so that differentiating primary, secondary and tertiary contents.\\

Consistency can be improved by defining a design system that dictates the guidelines for designing all the visual elements such as buttons, cards, drop-down menu and so on. In fact, at the moment the website totally lacks consistency regarding visual elements and this may lead users to be confused and perform the wrong action. Creating a design system would ensure consistency across the website and also an easier pattern recognition for users. Moreover, consistency should also be improved in the contents organisation of each page: when the section is the same, contents should be displayed in the same way in order to ensure a low cognitive effort.




