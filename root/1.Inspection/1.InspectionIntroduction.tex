\subsection{Introduction}

Inspection or experts' review is a structured technique in the field of usability evaluation that aims to assess some critical aspects and characteristics of a website to ensure its ease of use for the target users that might be interested in exploiting it.
This practice is carried out by experts (also called evaluators or inspectors), so people with a solid and robust knowledge around the principles and guidelines for the creation of an effective and pleasant product.

Among the several ways in which this practice can be organized, there is the so called heuristic-based methodology, which requires the various inspectors to establish in advanced a set of heuristics that will be the focus of the evaluation.\\
Thus, the steps that must be taken to complete an inspection procedure are the following. 
First off, since multiple inspectors have to collaborate and share their results, the initial phase can be considered as a setup of the actual inspection practice. Inspectors have to find an agreement on the general methodology that all of them must adopt to make the results comparable and useful for analysis. As a starting point, the evaluators must establish a set of heuristics that they will examine the website on. Of course, these heuristics have to be relevant for the website in question, as they must identify possible flaws or issues related to the website's usability. The definition of each heuristic should be unanimously agreed upon and noted down to rule out different subjective interpretations.\\
Part of this process is also to discuss among the inspectors the evaluation metrics, so a shared range of values (numerical or not numerical) that is suitable to assess the heuristics on the website. All the different values in the range should have a proper and unambiguous meaning which all inspectors are aware of before starting on the subsequent phases.

Once the setup is completed, inspectors can start working individually on the target website. At this stage, the practice consists in navigating on the website and trying to critically spot all the flaws, inconsistencies and errors that are present. To make the process organized, inspectors should focus on one heuristic at a time, leaving out the other ones momentarily. Once satisfied with the analysis of a single heuristic, the inspector assigns a score to the website for that heuristic and moves on to the next one.

The final step of inspection is the comparison and discussion of the results obtained by the evaluators. Ideally, all evaluators should meet, talk about the scores they assigned and find an agreement on a final shared mark to associate to each heuristic. This process is really a discussion, in which all experts argue about the reasons why they graded the website a certain way and eventually meet somewhere in the middle of their different ideas and opinions.

This chapter aims to show how the inspection has been performed on the \href{https://www.unicef.org/}{UNICEF website}, following the theoretical points illustrated in this introduction.
\clearpage

\subsection{Overview of UNICEF's website structure}

Information architecture focuses on organising information and defining the optimal structure for websites and mobile apps, helping users during the navigation to find and process the information needed.
For this reason, it is a key aspect when dealing with user experience online, as it should simplify the navigation of complex information for users. 
UNICEF's main menu in the website is organised under the header in five main sections that represent high-level pages, each with a quite clear intent that sufficiently matches users’ mental models.

\begin{itemize}
    \item \textbf{What we do:} this section contains UNICEF focus area and how they are addressing all the different topics.
    \item \textbf{Research and Reports:} this section contains reports and data provided by UNICEF to support and verify their work.
    \item \textbf{Stories:} this section contains stories of impact and information about specific emergencies to raise awareness and create empathy in the user.
    \item \textbf{About UNICEF:} this section delves into UNICEF's history, how the organisation is shaped and all the locations they are working in.
    \item \textbf{Take action:} this section is specifically for users interested in contributing to UNICEF's cause by working, partnership or volunteering.
\end{itemize}

As can be seen, it is not present a label for the homepage, but the logo in the left part of the header is clickable and allows the user to return to the homepage. The header also presents the button that allows users to donate for UNICEF and this option can be also found in the footer, that containts the link to the most important pages in the website.
Regarding these aspects, the website follows the standard organisation.

In addition to high-level pages, the website presents pages that can be reached through primary navigation, directly from the menu. 
More detailed pages (or lower level categories of the website's information architecture) that requires sub-navigation are a huge part of the website too, such as articles, blog posts and reports. To help users orientate themselves in the website, in some pages breadcrumbs are present, as well as links to more in-depth resources.

Due to the amount of information that the website collects and has to show, it is by its nature characterised by a complex structure that requires the user an initial effort in order to understand how to easily navigate among the pages. 
For this reason, the team took into consideration this intrinsic intricacy when performing the inspection, understanding the necessity of having an articulated information architecture.
\clearpage

\subsection{Team Methodology}
In order to perform the inspection, the team decided to have an initial meeting to navigate together the UNICEF International website, and thus having an initial shared background knowledge of the main strengths and weaknesses of the website. 
This was essential to speed up the analysis process, since the team already had a discussion about their professional opinions on the website.

To consolidate and inspect in depth all the different features of the website, it was deemed necessary to conduct an heuristic based analysis.
It was decided to analyse the website following:
\begin{enumerate}
    \item \textbf{Jakob Nielsen's 10 Usability Heuristics} to evaluate the overall usability and intuitiveness of the website.
    \item \textbf{MiLE+ methodology (Milano-Lugano Evaluation Method)}, focusing on the design dimensions of Navigation/Interaction, Content and Presentation and thus evaluating the quality of the design.
\end{enumerate}

At first, the team decided to perform the inspection of only a few pages that presented the majority of the problems or that collected different issues repeated along the website. The selected pages are the following: Homepage; What we do; Gender; Partnership; State of the World’s Children; Who we are; Donate; Volunteer; Where we work; Skills4Girls; Contact us.
However, by pursuing this approach, there was a risk of overlooking certain issues due to an analysis concentrated solely on specific sections of the site. For this reason, the team decided to keep these pages as a meaningful starting point to detect the most relevant and persistent issues, but with the freedom of navigating the website without constraints. In this way, it was possible to spot some hidden problems that, when met, could compromise the experience of users with the website.

Thereafter, the team established a set of numerical evaluation metrics to perform the inspection and a common inspection sheet, in order to carry out a consistent analysis and thus having a confrontation that addressed the strengths and weaknesses in a similar way.
Lastly, the team performed the inspection individually by setting a deadline. After the deadline, the team discussed together the results to define a final evaluation score, a comment and relevant examples for each heuristic.
\clearpage