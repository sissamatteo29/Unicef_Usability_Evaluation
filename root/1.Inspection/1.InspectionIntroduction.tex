\subsection{Introduction}

Inspection or experts' review is a structured technique in the field of usability evaluation that aims to assess some critical aspects and characteristics of a website to ensure its ease of use for the target users that might be interested in exploiting it.
This practice is carried out by experts, so people with a solid and robust knowledge around the principles and guidelines for the creation of an effective and pleasant website.

Among the several ways in which this practice can be organized, there is the so called heuristic-based methodology, which requires the various inspectors to establish in advanced a set of heuristics that will be the focus of the evaluation.\\
Thus, the steps that must be taken to complete an inspection procedure are the following. 
First off, since multiple inspectors have to collaborate and share their results, the initial phase can be considered as a setup of the actual inspection practice. Inspectors have to find an agreement on the general methodology that all of them must adopt to make the results comparable and useful for analysis. As a starting point, the evaluators must establish a set of heuristics that they will examine the website on. Of course, these heuristics have to be relevant for the website in question, as they must identify possible flaws or issues related to the usability of the website. The definition of each heuristic should be unanimously agreed upon and noted down to rule out different subjective interpretations.\\
Part of this process is also to discuss among the inspectors the evaluation metrics, so a shared range of values (numerical or not numerical) that is suitable to assess the heuristics on the website. All the different values in the range should have a proper and unambiguous meaning which all inspectors are aware of before starting on the subsequent phases.

All inspectors can now start working individually on the target website. At this stage, the practice consists in navigating on the website and trying to critically spot all the flaws, inconsistencies and errors that are present. To make the process organized, the inspector should focus on one heuristic at a time, leaving out the other ones momentarily. Once satisfied with the analysis of a single heuristic, the inspector assigns a score to the website for that heuristic and moves on to the next one.

The final step of inspection is the comparison and discussion of the results obtained by the evaluators. Ideally, all evaluators should meet, talk about the scores they assigned and find an agreement on a final shared mark to associate to each heuristic. This process is really a discussion, in which all experts argue about the reasons why they graded the website a certain way and eventually meet somewhere in the middle of their different ideas and opinions.

This chapter aims to show the inspection performed on the official Unicef website and all the points described above can be seen in the following.
\clearpage

\subsection{What is UNICEF}
UNICEF is an agency of the United Nations in charge of providing humanitarian and developmental aid to children worldwide. UNICEF International website is a collection of all the humanitarian actions addressed by the organisation in over 190 countries and territories, offering users the possibility to find in-depth information on various topics, as well as to donate for various causes.

Due to the amount of information that the website collects and has to show, it is by its nature characterised by a complex structure that requires the user an initial effort in order to understand to easily navigate among the pages. For this reason, the team took into consideration this intrinsic intricacy when performing the inspection, understanding the necessity of having an articulated information architecture.
\clearpage

\subsection{Team Methodology}
In order to perform the inspection, the team decided to have an initial meeting in order to navigate together the UNICEF International website, and thus having an initial shared background knowledge of the main strengths and weaknesses of the website. 
Subsequently, it was decided to analyse the website following Jakob Nielsen's 10 usability heuristics and MiLE+ methodology (Milano-Lugano Evaluation method), focusing on the design dimensions of Navigation/Interaction, Content and Presentation.

At first, the team decided to perform the inspection of only a few pages that presented the majority of the problems or that collected different issues repeated along the website. The selected pages are the following: Homepage; What we do; Gender; Partnership; State of the World’s Children; Who we are; Donate; Volunteer; Where we work; Skills4Girls; Contact us.
However, by pursuing this approach, there was a risk of overlooking certain issues due to an analysis concentrated solely on specific sections of the site. For this reason, the team decided to keep these pages as a meaningful starting point to detect the most relevant and persistent issues, but with the freedom of navigating the website without constraints.

Thereafter, the team established a set of numerical evaluation metrics to perform the inspection and a common inspection sheet, in order to carry out a consistent analysis.
Lastly, the team performed the inspection individually and then discussed together the results to define a final evaluation score, a comment and relevant examples for each heuristic.
\clearpage