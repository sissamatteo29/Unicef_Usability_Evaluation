
\subsection{Heuristics}
The set of heuristics chosen to perform the evaluation are Jakob Nielsen's 10 usability heuristics and MiLE+ methodology, solely taking into consideration the three dimensions of Navigation/Interaction, Content and Presentation. All the definitions of heuristics were interpreted with a positive logic and they can be read in the following paragraphs.

\subsubsection{Nielsen’s Heuristics}
Jakob Nielsen’s 10 usability heuristics were proposed in 1990 and they represent indispensable principles for designers in order to provide to users an interactive experience that is both intuitive and effortless.
\textit{Source: https://www.nngroup.com/articles/ten-usability-heuristics/}

\begin{itemize}
\item \textbf{H1. Visibility of System Status:} The design should always keep users informed about what is going on, through appropriate feedback within a reasonable amount of time.
\item \textbf{H2. Match Between the System and the Real World:} The design should speak the users' language. Use words, phrases, and concepts familiar to the user, rather than internal jargon. Follow real-world conventions, making information appear in a natural and logical order.
\item \textbf{H3. User Control and Freedom:} Users often perform actions by mistake. They need a clearly marked "emergency exit" to leave the unwanted action without having to go through an extended process.
\item \textbf{H4. Consistency and Standards:} Users should not have to wonder whether different words, situations, or actions mean the same thing. Follow platform and industry conventions.
\item \textbf{H5. Error Prevention:} Good error messages are important, but the best designs carefully prevent problems from occurring in the first place. Either eliminate error-prone conditions, or check for them and present users with a confirmation option before they commit to the action.
\item \textbf{H6. Recognition Rather than Recall:} Minimize the user's memory load by making elements, actions, and options visible. The user should not have to remember information from one part of the interface to another. Information required to use the design (e.g. field labels or menu items) should be visible or easily retrievable when needed.
\item \textbf{H7. Flexibility and Efficiency of Use:} Shortcuts — hidden from novice users — may speed up the interaction for the expert user so that the design can cater to both inexperienced and experienced users. Allow users to tailor frequent actions.
\item \textbf{H8. Aesthetic and Minimalist Design:} Interfaces should not contain information that is irrelevant or rarely needed. Every extra unit of information in an interface competes with the relevant units of information and diminishes their relative visibility.
\item \textbf{H9. Help Users Recognize, Diagnose, and Recover from Errors:} Error messages should be expressed in plain language (no error codes), precisely indicate the problem, and constructively suggest a solution.
\item \textbf{H10. Help and Documentation:} It’s best if the system doesn’t need any additional explanation. However, it may be necessary to provide documentation to help users understand how to complete their tasks.
\end{itemize}

\subsubsection{MiLE+ Method}
MiLE+ method is an experience-based usability evaluation framework for web applications that aims at providing a balanced approach between heuristic evaluation and task-driven techniques. The team focused on evaluating the quality of the design exploiting the set of heuristics collected into the design dimensions of Navigation/Interaction, Content and Presentation.
\textit{Source: https://docplayer.net/20989431-Mile-milano-lugano-evaluation-method.html}

\textbf{Content}
\begin{itemize}
    \item \textbf{H11. Information Overload:} Is the information in a page too much or too little?
    \item \textbf{H12. Consistency of Page Content Structure:} Do pages that present topics of the same category have the same types of elements?
    \item \textbf{H13. Contextualized Information:} Does the page include information that helps users understand where they are?
    \item \textbf{H14. Content Organisation (Hierarchy):} Is the hierarchical organisation of topics appropriate for the topic relevance?
\end{itemize}

\textbf{Navigation/Interaction}
\begin{itemize}
    \item \textbf{H15.  Interaction Consistency:} Do pages of the same type have the same navigation links and interaction capability?
    \item \textbf{H16.  Group Navigation - 1:} Is it easy to navigate from, among groups of “items”, and within the items?
    \item \textbf{H17.  Group Navigation - 2:} Do menus create cognitive overload?
    \item \textbf{H18.  Structural Navigation:} Is it easy to navigate among the “components” (“parts”) of a topic?
    \item \textbf{H19.  Semantic Navigation:} Is it easy to navigate from a topic to a related one (in both directions)?
    \item \textbf{H20.  Landmarks:} Are “landmarks” effective for the user to reach the “key” (most relevant) parts of the web site?
\end{itemize}

\textbf{Presentation}
\begin{itemize}
    \item \textbf{H21.  Text Layout:} Is the text readable? Is the font size appropriate?
    \item \textbf{H22.  Interaction Placeholders - Semiotics:} Are interactive elements “intuitive”?
    \item \textbf{H23.  Interaction Placeholders - Consistency:} Are textual or visual labels of interactive elements consistent in terms of wording, shape, colour, position, etc.?
    \item \textbf{H24.  Consistency of Visual Elements:} In pages of the same type do visual elements have the same visual properties?
    \item \textbf{H25.  Hierarchy - 1:} Is the on-screen allocation of contents within a page appropriate for their relevance?
    \item \textbf{H26.  Hierarchy - 2:} Is the on-screen allocation of visual elements appropriate for their relevance?
    \item \textbf{H27.  Spatial Allocation - 1:} Are “semantically related” elements close to each other?
    \item \textbf{H28.  Spatial Allocation - 2:} Are “semantically distant” elements placed distant from each other?
    \item \textbf{H29.  Consistency of Page Spatial Structure:} Do pages of the same type have the same spatial organisation for the various visual elements?
\end{itemize}
