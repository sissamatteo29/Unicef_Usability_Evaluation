
\subsection{Metrics}

To perform a consistent analysis, it was necessary to define a set of numerical metrics to evaluate each heuristic.
Initially, the team engaged in a discussion about the best scale to use and the relative qualitative value. The team was undecided whether to use a scale of 0 to 5 or 1 to 5 and about which kind of definition to assign to each number.
The final decision fell on using a numerical scale of 1 to 5 in order to avoid overly neutral judgments, as well as setting a clear difference between each number of the scale in terms of qualitative definition. 

Since all the definitions of heuristics were interpreted with a positive approach, the team defined the following evaluation metrics scale.
\begin{itemize}
    \item 1: Totally insufficient / Totally unsatisfied
    \item 2: Mildly wrong
    \item 3: Acceptable
    \item 4: Minor problems / Almost correct
    \item 5: Totally correct / Totally satisfied
\end{itemize}

To perform the individual inspection easily, the team created a common inspection sheet where each evaluator could assign the value to each heuristic, add a comment and meaningful examples. Having a shared methodology to perform the analysis was crucial for the team to perform a coherent inspection, but most importantly to facilitate the discussion and the final assignment of scores.