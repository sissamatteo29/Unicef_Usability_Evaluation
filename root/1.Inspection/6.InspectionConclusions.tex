
\subsection{Conclusions and considerations}

The heuristic inspection supported the team in better defining and understanding the most impactful weaknesses of the website. In fact, despite its vastness, the website presents the same type of issues among the different pages and thus it was possible to embed the concerns in the different heuristic categories. Moreover, by extending MiLE+ categories to Nielsen’s heuristics, it was possible to draw more accurate and comprehensive conclusions that took into consideration the design dimensions of Content, Navigation/Interaction and Presentation.
UNICEF’s website is overall characterised by a complex information architecture that together with inconsistencies and a poor hierarchy prevent the user from having a fluid and pleasant navigation.\\

Regarding the design dimension of Content, the mean of the scores is 2.5 on 5, demonstrating how the website clearly presents significant issues that fall into this category that can compromise in a meaningful way the interaction, based on the gravity of the issue itself. This category includes the following heuristics: H4; H6; H8; H10 and H11 - H14.
Inconsistency is one of the main concerns present in the website and it affects many different aspects, starting from the visual design of elements, such as buttons and cards which present ambiguous meanings, labelling and colours. Inconsistency is also present when it comes to pages that should present the same kind of topic, where inconsistent linkages between related topics affects the navigation. The lack of consistency prevents users from easily recognizing patterns and thus can lead to abandoning the website, due to confusion and frustration.
In addition, UNICEF's website is unnecessarily cluttered with information and visual elements that do not support the user in finding the key information, often resulting in having difficulties in accomplishing the goal. Since a proper hierarchy is missing, the user experience is shaped by repetitions and information overload resulting in a high chance of making mistakes during the navigation.\\

The category of Navigation/Interaction scored a mean of 2.923076923 on 5, meaning that the website presents inefficiencies that do not, however, seriously affect usability. This category includes the following heuristics: H1; H3; H4; H5; H6; H7; H9 and H15 - H20.
Since UNICEF’s website is very complex and presents many different information, it would be essential to always keep the user informed about his or her position in the website. However, the lack of breadcrumbs across the whole website (they are only present in a few pages) highly impact this aspect, making the user feel lost and not able to recognize the system status. 
The first level navigation is quite intuitive and clear, presenting an organised and appropriate menu, even though some unnecessary repetitions in the architecture are present. Instead, second and third level navigation can require more cognitive effort due the presence of minor discrepancies, making it challenging for users to understand their location within the site. This inconsistency can lead to confusion, especially when users are redirected to similar-looking external websites without clear differentiation. Moreover, inconsistency of interactive elements and the lack of hierarchy may hinder users' ability to prioritise information and understand its relevance.
Regarding efficiency, most of the time mistakes are correctly highlighted and landmarks are present and provide quick access to the main parts of the website. Nevertheless, in some cases landmarks are difficult to find due to their reduced size, underlining once again the serious hierarchy problems the website presents.\\

With regard to the Presentation category, the mean of the scores is 3.0625 on 5, meaning that the issues encountered negatively impact the user experience but they do not prevent the user from accomplishing his or her goals. This category includes the following heuristics: H2; H3; H4; H5; H6; H8; H9 and H21 - H29.
The main issue that affects this category is the one of consistency, as explained above. This leads to a sense of confusion and disorientation when navigating the website, since even different contents are usually presented in the same way. Together with inconsistency, the missing of a proper hierarchy in terms of dimensions of contents (both textual and visual) compromises an effective understanding of the most relevant information. Priority and relevance is not given to different contents and this aspect, together with an overload of information, may result in annoyance and discouragement.
Regarding the structure of the pages, generally they follow the same logic with minor discrepancies and this is valid also for the language used across the website, that usually matches the users’ mental models. Semantic relationships are displayed in an effective way, but a higher focus on proper dimensions of contents would improve this aspect a lot.\\

In conclusion, UNICEF’s website is acceptable in some respects and being mainly for consulting information, doesn’t offer many interactions that could lead to error-prone situations. However, concerns connected to hierarchy, consistency and information overload have a high risk of tragically compromising user interaction. Due to feelings such as exasperation, disappointment, uncertainty and disorientation, users may decide to close the website without getting to know more about UNICEF’s work, as well as choose to not donate because of insecurity. Being the website a crucial touchpoint to build knowledge about UNICEF’s work, a clear, consistent and engaging structure is necessary to attract users and raise awareness towards contemporary topics.


