
\subsubsection{Evaluator 4 user testing scores}

\subsubsection*{User testing record tables}
The following is a list of tables representing the most relevant quantitative and qualitative data collected during the user testing practice by the fourth evaluator.
Along with each table there is a section dedicated to comments and observations that the inspector managed to collect while running the tests on the user.

\vspace{0.8cm}

{
	\centering
	\renewcommand{\arraystretch}{1.2}
	\begin{minipage}{\textwidth}
		
		\centering	{ \large \textbf{User n°1}}
		\vspace{0.3cm}
		
		\begin{tabularx}{\textwidth}{|*{4}{>{\centering\arraybackslash}X|} >{\centering\arraybackslash}p{2.2cm}| >{\centering\arraybackslash}p{2.2cm}|}
			\hline
			\nohyphens{\textbf{Task Number}}& \textbf{Start Time} & \textbf{End Time} & \textbf{Elapsed time} & \nohyphens{ \textbf{Task Completion}} & \textbf{Task difficulty} \\ \hline
			1 & 16:05 & 16:09 & 4min & Failure & 4 \\ \hline
			2 & 16:10 & 16:12 & 2min & Completed & 1 \\ \hline
			3 & 16:13 & 16:15 & 2min & Completed & 1 \\ \hline
			4 & 16:16 & 16:17 & 1min & Completed & 1 \\ \hline
			5 & 16:18 & 16:21 & 3min & Completed & 3 \\ \hline
			6 & 16:22 & 16:27 & 5min & Completed ALT & 4 \\ \hline
		\end{tabularx}
		
		\vspace{0.7cm}
	\end{minipage}
}
\noindent
{\large \textbf{Comments and observations}}
% Space for comments here
\begin{enumerate}
	\item \textbf{Task 1}: The user starts by searching articles in the main page of the website, and then navigates to the section \href{https://www.unicef.org/reports}{Publication by topic}. He gets stuck there trying to apply various filter. He surrenders at the end, so the first task is considered as failed.
	\item \textbf{Task 2}: The user first tries with the Donate button in the top right corner of the website. When he realizes that it leads to a general donation for UNICEF, he goes back and finds the right section for supporting the Afghanistan's crisis specifically. At the end he complains about the position of this section under "Stories", which he finds counter intuitive.
	\item \textbf{Task 3}: The user easily senses that the correct section is under "About UNICEF" in the navbar and then clicks on \href{https://www.unicef.org/where-we-work#europe-and-central-asia}{Europe and Central Asia}. He complains about the fact that the drop-down menu on this page doesn't show Italy among the options, so he rolls back to the home page and finds the correct path under "All locations".
	\item \textbf{Task 4}: The user finds the correct path for the \href{https://www.unicef.org/reports/state-of-worlds-children}{report} immediately. As he scrolls the presentation page, he shows confusion for the length of the page and also for the example stories that are illustrated at the very beginning. He does also comment positively the animations that are placed on this page. 
	\item \textbf{Task 5}: The user seems confident about being able to find the right section without too much effort as he's starting to adapting and getting familiar with the website structure. 
	\item \textbf{Task 6}: As a first attempt, the user tries to navigate to the page \href{https://www.unicef.org/reports}{Publication by topic} and uses some filters there to spot the article requested by the task. After a lot of failing attempts, he resorts to the search box in the top right area of the website and completes the task.
	
\end{enumerate}





\vspace{1cm}

{
	\centering
	\renewcommand{\arraystretch}{1.2}
	\begin{minipage}{\textwidth}
		
		\centering	{ \large \textbf{User n°2}}
		\vspace{0.3cm}
		
		\begin{tabularx}{\textwidth}{|*{4}{>{\centering\arraybackslash}X|} >{\centering\arraybackslash}p{2.2cm}| >{\centering\arraybackslash}p{2.2cm}|}
			\hline
			\nohyphens{\textbf{Task Number}}& \textbf{Start Time} & \textbf{End Time} & \textbf{Elapsed time} & \nohyphens{ \textbf{Task Completion}} & \textbf{Task difficulty} \\ \hline
			1 & 20:41 & 20:43 & 2min & Completed & 1 \\ \hline
			2 & 20:44 & 20:46 & 2min & Completed & 1 \\ \hline
			3 & 20:47 & 20:50 & 3min & Completed & 3 \\ \hline
			4 & 20:51 & 20:55 & 4min & Completed & 2 \\ \hline
			5 & 20:56 & 20:58 & 2min & Completed & 2 \\ \hline
			6 & 20:58 & 21:04 & 6min & Completed ASSISTED & 4 \\ \hline
		\end{tabularx}
		
		\vspace{0.7cm}
	\end{minipage}
}
\noindent
{\large \textbf{Comments and observations}}
\begin{enumerate}
	\item \textbf{Task 1}: The user immediately spots the right section of the website to use and finds the articles without too much effort. He complains about the page \href{https://www.unicef.org/nutrition}{Nutrition} saying that it's less clear than the other ones.
	\item \textbf{Task 2}: The user first tries with the Donate button in the top right corner of the website. When he realizes that it leads to a general donation for UNICEF, he goes back and finds the right section for supporting the Afghanistan's crisis specifically. At the end he complains about the position of this section under "Stories", which he finds counter intuitive.
	\item \textbf{Task 3}: The user easily senses that the correct section is under "About UNICEF" in the navbar and then clicks on \href{https://www.unicef.org/where-we-work#europe-and-central-asia}{Europe and Central Asia}. He complains about the fact that the drop-down menu on this page doesn't show Italy among the options, so he rolls back to the home page and finds the correct path under \href{https://www.unicef.org/where-we-work}{All locations}. Finally, he also complains about the font size, which he finds too small.
	\item \textbf{Task 4}: First off, the user complains about the ambiguity between the two very close sections \textit{The state of the world's children} and \textit{Annual report}, which might have some overlapping meaning, since \textit{The state of the world's children} is an annual report. As the user navigates on the page for the \textit{The state of the world's children}, he shows confusion for the variety of content presented on the page.
	\item \textbf{Task 5}: The user seems to find the process for applying for jobs or internships very straightforward and intuitive. The only thing he complains about is the fact that in the search page for jobs there is no possibility to type text in the filter boxes to speed up the search of specific filter elements (e.g. the country in which the internship is looked for).
	\item \textbf{Task 6}: The user gets stuck in the page \href{https://www.unicef.org/blog}{UNICEF Blog} as he starts going through all the articles presented there. After a while, a hint is provided to help him reach the correct section of the website.
	
\end{enumerate}





\vspace{1cm}

{
	\centering
	\renewcommand{\arraystretch}{1.2}
	\begin{minipage}{\textwidth}
		
		\centering	{ \large \textbf{User n°3}}
		\vspace{0.3cm}
		
		\begin{tabularx}{\textwidth}{|*{4}{>{\centering\arraybackslash}X|} >{\centering\arraybackslash}p{2.2cm}| >{\centering\arraybackslash}p{2.2cm}|}
			\hline
			\nohyphens{\textbf{Task Number}}& \textbf{Start Time} & \textbf{End Time} & \textbf{Elapsed time} & \nohyphens{ \textbf{Task Completion}} & \textbf{Task difficulty} \\ \hline
			1 & 19:07 & 19:16 & 9min & Completed & 4 \\ \hline
			2 & 19:17 & 19:24 & 7min & Completed ASSISTED & 3 \\ \hline
			3 & 19:25 & 19:29 & 4min & Completed & 2 \\ \hline
			4 & 19:29 & 19:33& 4min & Completed & 2 \\ \hline
			5 & 19:34 & 19:42 & 8min & Failure & 4 \\ \hline
			6 & 19:43 & 19:45 & 2min & Completed & 1 \\ \hline
		\end{tabularx}
		
		\vspace{0.7cm}
	\end{minipage}
}
\noindent
{\large \textbf{Comments and observations}}
\begin{enumerate}
	\item \textbf{Task 1}: The user has a hard time understanding the pattern in the task, meaning that she doesn't recognize the fact that all the requested pages are under the same section of the website "All areas". This also shows a flaw in the user's adaptation process for what concerns the way pages are organized in the navigation bar.
	\item \textbf{Task 2}: As a first attempt, the user tries donating with the generic "Donate" button placed in the top right area of the web page. She tries in all possible ways to change the parameters for the donation from this point of the website with no success and seems very frustrated about that. Given a little hint on the fact that there is a dedicated page to the Afghanistan's crisis, she manages to pull the task off.
	\item \textbf{Task 3}: The user tries to use the search box of the website to find a way to the Italian site, with no success. Then she goes back to the home page and finds the correct path to solve the task.
	\item \textbf{Task 4}: The user seems to find the way to the report immediately and with a certain confidence. Moreover, she exploits the links placed inside the text of the page to jump immediately to the "Solutions" section. She positively comments about the fact that these links at the beginning of the text were helpful. As a final note, the user reads the content of the "Solutions" section and seems confused about the bullet points requested by the task. This confusion is due to the fact that the points presented on the page are interleaved with images and other real stories that disorient the user.
	\item \textbf{Task 5}: The user finds the correct section of the website under \href{https://www.unicef.org/careers/}{Work with us}, but she doesn't click immediately on the button presented on the page to look for jobs. Instead she navigates to the page \href{https://www.unicef.org/careers/explore-careers-unicef}{Explore Careers} and gets stuck there trying to find a solution. After showing frustration for not being able to solve the task, she decides to give up on the job.
	\item \textbf{Task 6}: The user has immediately the intuition to navigate to the presentation page dedicated to \href{https://www.unicef.org/coronavirus/covid-19}{COVID-19} and finds the article quite easily.
	
\end{enumerate}






\vspace{1cm}

{
	\centering
	\renewcommand{\arraystretch}{1.2}
	\begin{minipage}{\textwidth}
		
		\centering	{ \large \textbf{User n°4}}
		\vspace{0.3cm}
		
		\begin{tabularx}{\textwidth}{|*{4}{>{\centering\arraybackslash}X|} >{\centering\arraybackslash}p{2.2cm}| >{\centering\arraybackslash}p{2.2cm}|}
			\hline
			\nohyphens{\textbf{Task Number}}& \textbf{Start Time} & \textbf{End Time} & \textbf{Elapsed time} & \nohyphens{ \textbf{Task Completion}} & \textbf{Task difficulty} \\ \hline
			1 & 20:05 & 20:09 & 4min & Completed & 2 \\ \hline
			2 & 20:10 & 20:12 & 2min & Completed & 1 \\ \hline
			3 & 20:13 & 20:14 & 1min & Completed & 1 \\ \hline
			4 & 20:15 & 20:17 & 2min & Completed & 1 \\ \hline
			5 & 20:18 & 20:20 & 2min & Completed & 1 \\ \hline
			6 & 20:21 & 20:27 & 6min & Completed ALT & 4 \\ \hline
		\end{tabularx}
		
		\vspace{0.7cm}
	\end{minipage}
}
\noindent
{\large \textbf{Comments and observations}}
\begin{enumerate}
	\item \textbf{Task 1}: The user immediately finds the correct section for the page on \href{https://www.unicef.org/gender-equality}{Gender Equality}, but then has a hard time repeating the same procedure also for the other similar pages, since the sections dedicated to resources are different.
	\item \textbf{Task 2}: The user doesn't even need to click on the general Donate button to guess that the button leads to a general donation to UNICEF and not to a specific contribution in favour of the Afghanistan crisis.
	\item \textbf{Task 3}: The user complains about the layout of the page and the way countries are organised under the letters of the alphabet.
	\item \textbf{Task 4}: The user has a hard time finding the correct section inside the text, he scrolls up and down the report page a few times and complains about the length of the page which doesn't help him having a clear high-level view of the content.
	\item \textbf{Task 5}: The user seems to be very confident in the search process. He only complains about the fact that there are multiple ways in the search section for jobs to set the filter for searching an "internship".
	\item \textbf{Task 6}: The user gets stuck for a lot of time inside the page \href{https://data.unicef.org/?_gl=1\%2A1he5ywc\%2A_ga\%2AMTEzMTU1MTkxOS4xNzEwMTk2NDI2\%2A_ga_ZEPV2PX419\%2AMTcxMjYxMDM0Ni4yNi4xLjE3MTI2MTUzNDEuNjAuMC4w}{Data by topic and country}. After providing a hint on the position of the article, the user manages to complete the task but he also complains about the the fact that there is no real internal organisation of articles inside the website.
	
\end{enumerate}



\vspace{1cm}

{
	\centering
	\renewcommand{\arraystretch}{1.2}
	\begin{minipage}{\textwidth}
		
		\centering	{ \large \textbf{User n°5}}
		\vspace{0.3cm}
		
		\begin{tabularx}{\textwidth}{|*{4}{>{\centering\arraybackslash}X|} >{\centering\arraybackslash}p{2.2cm}| >{\centering\arraybackslash}p{2.2cm}|}
			\hline
			\nohyphens{\textbf{Task Number}}& \textbf{Start Time} & \textbf{End Time} & \textbf{Elapsed time} & \nohyphens{ \textbf{Task Completion}} & \textbf{Task difficulty} \\ \hline
			1 & 16:10 & 16:14 & 4min & Completed & 2 \\ \hline
			2 & 16:15 & 16:25 & 10min & Completed ASSISTED & 4 \\ \hline
			3 & 16:26 & 16:29 & 3min & Completed ALT & 1 \\ \hline
			4 & 16:30 & 16:39 & 9min & Completed & 3 \\ \hline
			5 & 16:40 & 16:45 & 5min & Completed & 2 \\ \hline
			6 & 16:46 & 16:55 & 9min & Failure & 4 \\ \hline
		\end{tabularx}
		
		\vspace{0.7cm}
	\end{minipage}
}
\noindent
{\large \textbf{Comments and observations}}
\begin{enumerate}
	\item \textbf{Task 1}: The user comments negatively the way articles are organised under the page for \href{https://www.unicef.org/gender-equality}{Gender Equality}, but he manages to complete the task quite easily and without showing too much cognitive effort.
	\item \textbf{Task 2}: The user tries to find the correct path on the website by searching in the "Where we work" section. In here, he manages to find the descriptive page about what UNICEF is doing in Afghanistan, but all the buttons to donate from here lead to the general donation to UNICEF, not to the support for the Afghanistan crisis. After getting lost in the Afghanistan page for quite a while, the user is helped with an hint and he manages to find the correct section eventually.
	\item \textbf{Task 3}: The user first tries to browse the page dedicated to \href{https://www.unicef.org/partnerships}{Partnerships}, with no results. Then, he manages to find the website dedicated to UNICEF Italy with an alternative path (compared to the expected one) through the webpage \href{https://www.unicef.org/eca/}{Europe and Central Asia}. He comments negatively about the fact that the layout and aesthetic of UNICEF Italy's page is completely different from the global one.
	\item \textbf{Task 4}: The user wastes a lot of time on the page \href{https://www.unicef.org/reports/unicef-annual-report-2022}{Annual Report} as he doesn't spot the correct section right away. He also complains about how this \href{https://www.unicef.org/reports/unicef-annual-report-2022}{Annual Report} page is structured, especially he finds the images in the background (while scrolling down) very disturbing for the reading. The user also tries the section \href{https://www.unicef.org/reports}{Publication by topic} before finding the correct answer to the task.
	\item \textbf{Task 5}: The user intuitively navigates under the "Take action" section, but he immediately complains about the ambiguity between the titles for the subsections \href {https://www.unicef.org/careers/}{Work with us} and \href{https://www.unicef.org/partnerships}{Partner with us}, which might be interpreted as overlapping.
	\item \textbf{Task 6}: The user tries many possible paths inside the website without succeeding in any of them. In particular, he browses all the following three web pages: \href{https://data.unicef.org/?_gl=1\%2A1he5ywc\%2A_ga\%2AMTEzMTU1MTkxOS4xNzEwMTk2NDI2\%2A_ga_ZEPV2PX419\%2AMTcxMjYxMDM0Ni4yNi4xLjE3MTI2MTUzNDEuNjAuMC4w}{Data by topic and country}, \href{https://www.unicef.org/blog}{UNICEF Blog}, \href{https://www.unicef.org/reports}{Publications by topic}.
	
\end{enumerate}

\clearpage




\subsubsection*{Post-test questionnaire results}
This section is dedicated to a summary table that helps illustrating how the users responded to the post-test questionnaire they've been required to fill in.\\

{
	\renewcommand{\arraystretch}{1.2}
	\centering
	\begin{tabularx}{\textwidth}{|*{6}{>{\centering\arraybackslash}X|}}
		\hline
		\multirow{2}{*}{\textbf{Question}} & \multicolumn{5}{c|}{\textbf{Participant ID}} \\ \cline{2-6}
		& 1 & 2 & 3 & 4 & 5 \\ \hline
		1 & D & A & A & A & SD \\ \hline
		2 & SD & SD & N & A & D \\ \hline
		3 & SD & D & N & N & D \\ \hline
		4 & SA & SA & A & A & SA \\ \hline
		5 & SA & A & N & A & A \\ \hline
		6 & A & D & A & A & SD \\ \hline
		7 & N & SA & D & N & A \\ \hline
		8 & SA & D & N & A & A \\ \hline
		9 & D & A & N & D & D \\ \hline
		10 & SD & D & D & D & D \\ \hline
		11 & A & N & D & D & D \\ \hline
		12 & A & SD & N & D & SD \\ \hline
		13 & SD & D & N & D & D \\ \hline
		14 & D & D & D & D & D \\ \hline
		15 & N & D & N & SA & D \\ \hline
		16 & A & SA & N & A & D \\ \hline
		17 & A & SA & A & A & A \\ \hline
		18 & N & SA & A & A & A \\ \hline
	\end{tabularx}
}


\textbf{Legend:} SA = Strongly Agree, A = Agree, N = Neutral, D = Disagree, SD = Strongly Disagree

\vspace{0.5cm}

\textbf{Adjectives per participant:}
\begin{enumerate}
	\item Huge, disorganised, aesthetically pleasant
	\item Disorganised, unclear, badly-designed
	\item Confusing, colourful, overwhelming
	\item Disorganised, unintuitive, mediocre
	\item Unclear, badly-structured, interesting
\end{enumerate}

\vspace{0.25cm}

\textbf{Additional comments per participant:}
\begin{enumerate}
	\item The user says that he didn't manage to fully understand the structural organisation of the website even after completing all the tasks required.
	\item Since the user studies design, he claims that the website should be more consistent in terms of structure inside the pages and also icons used throughout the website.
	\item The user is particularly interested in some of the topics covered by the website like climate change and gender equality, so she would like the website to have a better organisation of the articles and posts to use it more frequently and in a more convenient way.
	\item No comments.
	\item The user finds the topics treated by the website very relevant and interesting, so he feels like they should be presented in a more minimal and effective way to have a better impact on today's society.
\end{enumerate}