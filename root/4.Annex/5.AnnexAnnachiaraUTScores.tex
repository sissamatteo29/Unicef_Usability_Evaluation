
\subsection{Evaluator 1 user testing scores}

\subsubsection*{User testing record tables}
The following is a list of tables representing the most relevant quantitative and qualitative data collected during the user testing practice by the first evaluator.
Along with each table there is a section dedicated to comments and observations that the inspector managed to collect while running the tests on the user.

\vspace{0.8cm}

{
	\centering
	\renewcommand{\arraystretch}{1.2}
	\begin{minipage}{\textwidth}
	
 	\centering	{ \large \textbf{User n°1}}
	\vspace{0.3cm}
	
	\begin{tabularx}{\textwidth}{|*{4}{>{\centering\arraybackslash}X|} >{\centering\arraybackslash}p{2.2cm}| >{\centering\arraybackslash}p{2.2cm}|}
		\hline
		 \nohyphens{\textbf{Task Number}}& \textbf{Start Time} & \textbf{End Time} & \textbf{Elapsed time} & \nohyphens{ \textbf{Task Completion}} & \textbf{Task difficulty} \\ \hline
		1 & 12:35 & 12:38 & 03:47 & Completed & 3 \\ \hline
		2 & 12:42 & 12:49 & 07:18 & Failure & 5 \\ \hline
		3 & 12:51 & 12:51 & 00:33 & Completed ALT & 1 \\ \hline
		4 & 12:54 & 13:01 & 07:02 & Partial & 4 \\ \hline
		5 & 13:02 & 13:05 & 03:35 & Completed ALT & 2 \\ \hline
		6 & 13:11 & 13:14 & 03:47 & Completed & 4 \\ \hline
	\end{tabularx}
	
	\vspace{0.7cm}
	\end{minipage}
}
\noindent
{\large \textbf{Comments and observations}}
% Space for comments here
\begin{itemize}
    \item \textbf{Task 1:} She reachs "gender" quickly. At first, click on the index of the page, difficulty in finding the "Resources" button, same for Covid-19. Nutrion ok. She scrolls the pages up and down without finding the information quickly. She says that the website is not built in the same way for each section.
    \item \textbf{Task 2:} Click on "donate" in the header and expects to find there the option, then try to search for Afghanistan in the search bar, but can't find the option. Click on "become a donor" in the footer of Afghanistan page, try again in that page with the drop-down menu. Can't finish the task.
    \item \textbf{Task 3:} Intuitive, no errors. She selected directly "Europe and Central Asia" in the menu. She finds the huge letters strange, but the map is nice.
    \item \textbf{Task 4:} Initial path ok, until report page. Here, confused by the page, videos and images. She finds the correct section, but read the bulleted list. Huge images, no clear distinction of sections, no bold. Confusion. Can't find the correct 4 solutions.
    \item \textbf{Task 5:} Take action - Volunteers - Careers - Internship. From here, click on the right link and open right page but not intuitive filters (they don't seem selected and they should be near the filters selection). She did a longer path than expected.
    \item \textbf{Task 6:} She opens "Press center" and uses the search bar without results, tries with filters but no right results. Then, she follows the right path searching for covid in the menu. She is bewildered by not finding anything in the press centre, she expects articles and stuff here.
\end{itemize}

diocan





\vspace{1cm}

{
	\centering
	\renewcommand{\arraystretch}{1.2}
	\begin{minipage}{\textwidth}
		
		\centering	{ \large \textbf{User n°2}}
		\vspace{0.3cm}
		
		\begin{tabularx}{\textwidth}{|*{4}{>{\centering\arraybackslash}X|} >{\centering\arraybackslash}p{2.2cm}| >{\centering\arraybackslash}p{2.2cm}|}
			\hline
			\nohyphens{\textbf{Task Number}}& \textbf{Start Time} & \textbf{End Time} & \textbf{Elapsed time} & \nohyphens{ \textbf{Task Completion}} & \textbf{Task difficulty} \\ \hline
			1 & 21:43 & 21:44 & 01:50 & Completed & 3 \\ \hline
			2 & 21:46 & 21:47 & 01:20 & Completed ALT & 4 \\ \hline
			3 & 21:49 & 21:49 & 00:39 & Completed ALT & 5 \\ \hline
			4 & 21:52 & 21:53 & 01:28 & Completed & 4 \\ \hline
			5 & 21:55 & 21:57 & 02:14 & Completed ASSISTED & 3 \\ \hline
			6 & 21:59 & 22:00 & 01:58 & Completed ASSISTED & 2 \\ \hline
		\end{tabularx}
		
		\vspace{0.7cm}
	\end{minipage}
}
\noindent
{\large \textbf{Comments and observations}}
% Space for comments here
\begin{itemize}
	\item \textbf{Task 1:} Found the page and the button shortly, but for Covid is confused since there is not that option. Nutrion ok. He says he doesn't expect these topics to be in that section.
	\item \textbf{Task 2:} Click on "donate" in the header, read FAQ, open drop-down menu. Frustrate since he can't find the option. Then read all the menu, find Afghanistan in "stories" and complete task. He says it is not intuitive at all, he just searched for the name of the country.
	\item \textbf{Task 3:} About - Europe - Select country (can't find Italy) - Visit region site - Italy. He says he randomly searched for it, it is not very clear.
	\item \textbf{Task 4:} Report - The state of the world - He reades the page confused and angry at all that text. Read the first 4 bullet points and think they are those. Then indicate the correct ones. Can't understand the organisation of the page due to poor differentiation.
	\item \textbf{Task 5:} Take action - Volunteer - Partner - Work - Search jobs. He didn't used the filter for "contract" because didn't see it (I told him to do it), he says there is no hierarchy because usually "work with us" is highlighted.
	\item \textbf{Task 6:} What we do - Covid 19 - Ok. He can't find search bar, said it is not so much visible.
\end{itemize}


\vspace{1cm}

{
	\centering
	\renewcommand{\arraystretch}{1.2}
	\begin{minipage}{\textwidth}
		
		\centering	{ \large \textbf{User n°3}}
		\vspace{0.3cm}
		
		\begin{tabularx}{\textwidth}{|*{4}{>{\centering\arraybackslash}X|} >{\centering\arraybackslash}p{2.2cm}| >{\centering\arraybackslash}p{2.2cm}|}
			\hline
			\nohyphens{\textbf{Task Number}}& \textbf{Start Time} & \textbf{End Time} & \textbf{Elapsed time} & \nohyphens{ \textbf{Task Completion}} & \textbf{Task difficulty} \\ \hline
			1 & 13:33 & 13:37 & 04:01 & Completed ALT & 3 \\ \hline
			2 & 13:38 & 13:42 & 04:06 & Completed ASSISTED & 5 \\ \hline
			3 & 13:43 & 13:43 & 00:59 & Completed ALT & 4 \\ \hline
			4 & 13:45 & 13:47 & 02:44 & Completed ALT & 5 \\ \hline
			5 & 13:49 & 13:50 & 01:05 & Completed & 2 \\ \hline
			6 & 13:51 & 13:52 & 01:10 & Completed & 2 \\ \hline
		\end{tabularx}
		
		\vspace{0.7cm}
	\end{minipage}
}
\noindent
{\large \textbf{Comments and observations}}
% Space for comments here
\begin{itemize}
	\item \textbf{Task 1:} Report - About - Gender from homepage (icon) - "Resouces" button. Go back to homepage and search for covid, here recognizes the option is not there. Nutrion ok. Says the info is too much.
	\item \textbf{Task 2:} Open "donate" button in the header, but no option in the drop-down menu. About - Where we work - Afghanistan website but here the option in donate is not present. Then by accident open stories - Afghanistan and complete.
	\item \textbf{Task 3:} Where we work - Europe - drop-down menu but Italy is not there. Visit region site - find Italy with the map. She liked the map but the path was not intuitive.
	\item \textbf{Task 4:} Search bar - search with the name - open. Reads the bullet points, then scroll and understand the right titles. Confused by all the animations and big images.
	\item \textbf{Task 5:} The task was carried out with no problems following the right path. Said the filters were confusing and not very clear.
	\item \textbf{Task 6:} The task was carried out with no problems. Suggest to add the research in the topic pages, because there are many information and it's difficult to orientate.
\end{itemize}






\vspace{1cm}

{
	\centering
	\renewcommand{\arraystretch}{1.2}
	\begin{minipage}{\textwidth}
		
		\centering	{ \large \textbf{User n°4}}
		\vspace{0.3cm}
		
		\begin{tabularx}{\textwidth}{|*{4}{>{\centering\arraybackslash}X|} >{\centering\arraybackslash}p{2.2cm}| >{\centering\arraybackslash}p{2.2cm}|}
			\hline
			\nohyphens{\textbf{Task Number}}& \textbf{Start Time} & \textbf{End Time} & \textbf{Elapsed time} & \nohyphens{ \textbf{Task Completion}} & \textbf{Task difficulty} \\ \hline
			1 & 10:20 & 10:22 & 02:27 & Completed & 3 \\ \hline
			2 & 10:24 & 10:28 & 04:46 & Completed ALT & 5 \\ \hline
			3 & 10:30 & 10:31 & 01:12 & Completed & 2 \\ \hline
			4 & 10:32 & 10:35 & 03:12 & Completed & 4 \\ \hline
			5 & 10:36 & 10:37 & 01:07 & Completed & 3 \\ \hline
			6 & 10:39 & 10:40 & 01:58 & Completed & 2 \\ \hline
		\end{tabularx}
		
		\vspace{0.7cm}
	\end{minipage}
}
\noindent
{\large \textbf{Comments and observations}}
% Space for comments here
\begin{itemize}
	\item \textbf{Task 1:} Found the right page for gender, covid and nutrition. Indicated the right option, but unsure about it since all different. She read the word "resources" and recognized by it, but still confused.
	\item \textbf{Task 2:} Clicked on all the different "donate" button and frustrated for not finding the option for Afghanist anaywhere. Then searrched in all the menu, found the word "afghanistan" and entered the page. Says it was frustrating and strange.
	\item \textbf{Task 3:} The task was carried out with no problems, but she is confused about the menu with the big letters, said it's ugly and not very clear.
	\item \textbf{Task 4:} Followed the right path, but in the report page was confused but all the text. Didn't find at first the 4 right solutions due to big images that were distracting.
	\item \textbf{Task 5:} The task was carried out with no problems, but she had some troubles in using the filters and understanding if they were selected. Said the result was not highlighted as much as she expected.
	\item \textbf{Task 6:} Everything ok. Said the search bar was not so much visible.
\end{itemize}



\vspace{1cm}

{
	\centering
	\renewcommand{\arraystretch}{1.2}
	\begin{minipage}{\textwidth}
		
		\centering	{ \large \textbf{User n°5}}
		\vspace{0.3cm}
		
		\begin{tabularx}{\textwidth}{|*{4}{>{\centering\arraybackslash}X|} >{\centering\arraybackslash}p{2.2cm}| >{\centering\arraybackslash}p{2.2cm}|}
			\hline
			\nohyphens{\textbf{Task Number}}& \textbf{Start Time} & \textbf{End Time} & \textbf{Elapsed time} & \nohyphens{ \textbf{Task Completion}} & \textbf{Task difficulty} \\ \hline
			1 & 15:10 & 15:13 & 03:17 & Completed ALT & 3 \\ \hline
			2 & 15:15 & 15:23 & 08:46 & Failure & 5 \\ \hline
			3 & 15:27 & 15:29 & 02:03 & Completed & 3 \\ \hline
			4 & 15:30 & 15:33 & 03:37 & Completed & 4 \\ \hline
			5 & 15:35 & 15:36 & 01:27 & Completed ALT & 2 \\ \hline
			6 & 15:37 & 15:39 & 02:32 & Completed ASSISTED & 4 \\ \hline
		\end{tabularx}
		
		\vspace{0.7cm}
	\end{minipage}
}
\noindent
{\large \textbf{Comments and observations}}
% Space for comments here
\begin{itemize}
	\item \textbf{Task 1:} Troubles in finding the right option in the menu and tries from the homepage, but can't find it. Then, finds it and indicated the Resources correctly. Confused by the difference in the graphical aspect.
	\item \textbf{Task 2:} Can't finish the task. Clicked on all "donate" buttons he found, tried with the drop-down menu, but can't finish. Frustrated because he says it is an important option.
	\item \textbf{Task 3:} Where we work - Europe - Select "I" in the menu and choose Italy. Says the selection is not intuitive and the text is too small (country's names).
	\item \textbf{Task 4:} Report - the state of the world - at first can't find the option to open the report because of the scrolling. Very confused by the solutions, but after some scrolling he indicates them correctly.
	\item \textbf{Task 5:} Take action - work with us. Used the search bar instead of filters because he didn't see them at first.
	\item \textbf{Task 6:} At first confused on how to search the information, because he can't orientate in the website. Then, with suggestion he finds Covid-19 in the menu and scroll until he finds it, but still says it takes too much time.
\end{itemize}

\clearpage




\subsubsection*{Post-test questionnaire results}
This section is dedicated to a summary table that helps illustrating how the users responded to the post-test questionnaire they've been required to fill in.\\

{
\renewcommand{\arraystretch}{1.2}
\centering
	\begin{tabularx}{\textwidth}{|*{6}{>{\centering\arraybackslash}X|}}
		\hline
		\multirow{2}{*}{\textbf{Question}} & \multicolumn{5}{c|}{\textbf{Participant ID}} \\ \cline{2-6}
		& 1 & 2 & 3 & 4 & 5 \\ \hline
		1 & D & SD & D & D & D \\ \hline
		2 & N & D & SD & D & D \\ \hline
		3 & D & N & D & D & D \\ \hline
		4 & SA & SA & SA & A & A \\ \hline
		5 & A & SA & A & A & SA \\ \hline
		6 & SA & D & N & N & A \\ \hline
		7 & D & SA & SA & SA & A \\ \hline
		8 & A & N & N & D & D \\ \hline
		9 & D & D & A & N & A \\ \hline
		10 & SD & SD & N & D & D \\ \hline
		11 & D & N & SD & D & SD \\ \hline
		12 & A & SD & N & SD & N \\ \hline
		13 & D & SD & SD & A & D \\ \hline
		14 & D & SD & SD & D & SD \\ \hline
		15 & D & SD & A & D & A \\ \hline
		16 & A & A & D & A & N \\ \hline
		17 & SA & A & A & SA & A \\ \hline
		18 & A & A & A & A & A \\ \hline
	\end{tabularx}
}


\textbf{Legend:} SA = Strongly Agree, A = Agree, N = Neutral, D = Disagree, SD = Strongly Disagree

\vspace{0.5cm}

\textbf{Adjectives per participant:}
\begin{enumerate}
	\item Complicated, Frustrating, Unclear
	\item Confusing, Unnecessarily complex, Takes too long to do simple actions
	\item Tiring, Over-crowded, Unnecessarily dense
	\item Complex, Strange to use, Repetitive
	\item Full of information, Unintuitive, Boring
\end{enumerate}

\vspace{0.25cm}

\textbf{Additional comments per participant:}
\begin{enumerate}
	\item The site is unclear in my experience, finding some sections was frustrating and this in my opinion decreases the user's attention and willingness to complete their research. This was the case for me in task number 2, I was so fed up with the unintuitive nature of the site that if it wasn't a test I would have given up making the donation and this is a shame given the work UNICEF does. As a person with DSA the site is really unfriendly.
	\item As an ADHD person I have noticed that there is a vast amount of useless images on the site that are very distracting to the person using it. It is unintuitive and confusing.
	\item As I browsed the site I got nervous because I forgot where to find the information, or I couldn't find it at all. I understand that it has to contain a lot of information but this way it doesn't entice the user to search, more like it leads to leaving the site and not trying again.
	\item /
	\item During the test I often felt frustrated because it took too long to perform simple tasks, which should be quick and intuitive even if you are not so familiar with the site. The most important function, that of donating, is confusing and does not convey too much confidence to me.
\end{enumerate}