
\subsection{Conclusions and considerations}

The team conducted the user testing in order to confirm the results that came out from the heuristic inspection and to spot new issues and difficulties that testers encountered when navigating UNICEF’s website. The tasks were defined based on the main problems found during the inspection and the testing session was carried out with 20 people.\\

All the tasks have a high percentage of success (78\%; 82\%; 87\%; 92\%, 85\%, 82\%), meaning that the testers succeeded in completing the goal even though sometimes they followed a different path than the established one. Moreover, the average time spent on each task resulted to be consistent with the one predicted by the team before performing the testing and for some tasks higher than expected, in particular for the tasks number 2 and 6.
Besides quantitative indicators, the team decided to also focus on qualitative indicators that could provide more insights about the actual problems found during the navigation. For this reason, comments and observations were taken both from the perspective of the evaluators and from the ones of each tester, as well as annotating moments of disorientation, wandering and frustrations. The mix between quantitative and qualitative indicators, together with the post-test questionnaire, allowed the team to have a clearer understanding regarding the perspective of testers that never used UNICEF’s website before.\\

The categories addressed were: Usability, Content, Interaction and Navigation, Aesthetic, Organization and Consistency. In general, for each category the average number of category critics (n CAT critics) is higher than the number of category praises (n CAT Praises), representing 75,14\%. The Aesthetic category is the one that received less comments by the testers, but this is understandable since none of the testers had a design background. Consistency is another aspect that was not really addressed by many users, but this could also be due to the short time spent on task and the lack of a professional background that may have allowed them to recognise inconsistencies. However, testers mainly commented in a negative way regarding this category: whether aware or not, the lack of consistency through the website may have implied a higher cognitive effort to succeed in the task. One example is that testers expected different functions from the different “Donate” buttons, and they expressed confusion when all of them led to the same page. 
The categories of Interaction and Navigation, Organisation and Content were highly criticised by users. In particular, testers found difficulties in finding the content requested in the page due to the amount of information displayed. Many of them complained about the images being too big and thus distracting and confusing.\\
 
One key aspect that emerged from the user testing, and that during the heuristic inspection was slightly overlooked, is the problems connected to information architecture. Despite having a high percentage of success, testers often followed alternative paths to complete the task and made mistakes along the way. By observing testers performing the tasks, it was possible to understand that alternative paths and mistakes mainly depended on the difficulties in finding the correct page/section in the main menu. The amount of content, the repetitions and the lack of clear distinction between the various sections led users to get lost in the website and thus trying different options before completing the task. For sure, this aspect created a lot of frustration and stress in the testers, who had to make a greater effort to perform tasks that are considered to be common in a website such UNICEF. This is reflected in the fact that 73.23\% of comments and observations on the tasks are negative. In fact, many testers identified the website as complex, unclear, confusing and saturated with information. They also described it as informative and visually pleasing, but they were the minority.\\

In conclusion, it can be stated that the user testing was helpful for confirming the problems found during the inspection (inconsistency, hierarchy, overload of information) and to also identify the importance of improving the information architecture in order to facilitate the navigation and thus enhance the overall usability of the website.
