
\subsection{Introduction}

User testing can be defined as a usability evaluation methodology that involves directly the target users of an application and aims to analyse their interactions and behaviours when they engage with the product being tested.
This practice is a very relevant and valuable source of information for the creators and designers of a website, as it provides a direct access to the perspective of the final users of the application.\\
These users are very often completely unfamiliar with the website being analysed, so every error, mistake or reaction from their first usage of the product results into paramount data to be collected.

Of course, user testing is a structured methodology, so it is fundamental to define the steps that have to be taken when applying this procedure in order to obtain tangible results.\\
Firstly, before involving the end users, the experts in charge of carrying out this practice (later called inspectors or evaluators) have to design the procedure and adapt it to the specific website to be tested. 
So they have to discuss and find an agreement on the following points:
\begin{enumerate}
	\item \textbf{Description of the user profile(s)}, which is the high-level description of the most important characteristics that a user must have in order to be recruited for the test (i.e. age range, level of experience with technology, interests and hobbies in specific fields...). This point is extremely relevant because the test should target only users that will likely be interested in using the final product, so a good definition of the user profile(s) allows to narrow the search only to the market segment that is the most relevant for the application.
	\item \textbf{Definition of quantitative and qualitative variables to measure} during the tests. This also includes establishing the range of values that each variable can assume so that the evaluation is consistent and coherent among all evaluators. 
	\item \textbf{Design of the tasks} to be performed by the user on the website. Indeed, in order to observe the user's interaction with the application, evaluators leverage some pre-defined tasks that the user is asked to complete. This fosters the consistency of data collection, as all users perform the same jobs.
	\item \textbf{Fabrication of the materials for data collection}. All inspectors should use the same sheets, forms and questionnaire to test their users. These sheets can be organised in different ways based on the specific product under analysis.
	\item \textbf{Definition of the hardware and software settings for the test}. The environment in which the test is conducted is also impactful on the final results, so the evaluators should agree on some common characteristics of the environment that surrounds the users while they attempt to complete the tasks.
	
\end{enumerate}

After this setup phase, a \textit{pilot test} can be performed to verify the correctness of the choices made. The pilot test is a trial test on one or more users that will not count in the final results of the user testing practice, as it serve as a confirmation of the fact that the design of the test has been correctly sorted out.
If some errors or issues are encountered during the pilot test, inspectors have the chance of adjusting some elements of the design.

At this point, evaluators are ready to recruit real users based on the user profile(s) previously defined and run the tests on them following all the guidelines as strictly as possible for consistency and coherency.

In the following section, a more concrete description of the methodology and execution for the user testing applied on the \href{https://www.unicef.org/}{Unicef website} is provided to show the reader how these high-level concepts have been concretely employed in the target website of this report.